\subsection{Application Framework}

Most of the additions to the applications framework
classes are discussed elsewhere (e.g. \code{Statistics}, \Sec{secBackground}; \code{Psf}, \Sec{secPsf}),
but the most far-reaching of the changes, the new image API, deserves its own section.

\subsubsection{The Image Classes}
\label{secImageClasses}

In DC2 pixels in images and masks were managed using a toolkit from NASA-Ames, the ``Vision Workbench'' (\code{vw}).
Access to the pixels in an image was achieved by explicitly following a pointer to a \code{vw} class,
thereby exposing all of \code{vw} API, including any incompatible changes that might appear in
the future.

A further major problem with \code{vw} is that its concept of pixel access is restricted to a simple
STL-like iterator, whereas many astronomical algorithms require examination of the neighbourhood of
a pixel.

At the end of DC2 we therefore decided to rewrite all of our image classes to use \code{boost::gil}
(an image library originating at Adobe) which has a richer set of ways of accessing pixels;  in
order to preserve encapsulation, we added a thin wrapper layer above the raw \code{boost::gil} calls.
Another major goal of the rewrite was to make the code that manipulated the pixels of a single
\code{Image} identical to that that manipulates a \code{MaskedImage}'s pixels --- thereby enabling
generic programming of e.g. convolutions.

All of these goals have been achieved for DC3a.  The new APIs have proved to be convenient and
powerful, and the code generated is efficient,\footnote{at least with g++ versions $>= 4.2$} even when
manipulating images, masks, and variances simultaneously when processing \code{MaskedImage}s.

