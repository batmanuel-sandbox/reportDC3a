% Section 6.8: Association

\subsection{Association}

The association pipeline compares the difference sources from a pair of
exposures (as produced by the image processing and detection pipeline), to
a catalog of known objects (as produced by LSST data release processing) as
well as to predicted positions of moving objects. The comparison is spatial;
all difference-source/object pairs where the source and object are within a
configurable angular separation of each-other are found and stored, as are
all difference-source/moving-object pairs where the source falls inside the
N-sigma error ellipse of the objects predicted position. These associations
provide the context from which the (yet to be implemented) alert generation
pipeline will decide whether or not a difference source should be alerted on.
Difference sources that are not matched to an existing object are (under
certain conditions) used to create new objects: this avoids repeated alerts
for longer lived occurrences upon subsequent visits to the same region
of the sky.

\subsubsection{Architecture}

For DC3a, the association pipeline was updated to keep pace with developments
in the underlying middleware and applications framework classes. However, its
architecture remains fundamentally the same as in DC2. We refer the interested
reader to the DC2 report for a more detailed discussion thereof, but provide a
brief outline below.

The association pipeline is organized into three main phases: a load phase, a
compute phase, and a store phase. The first of these loads objects within the
field-of-view being observed from the file-system into memory. On disk,
objects are organized into read-only chunk files, obtained by dividing objects
into equal height declination stripes, and then physically partitioning these
into chunks of equal width in right ascension. Each chunk file has an
associated chunk-delta file which records new objects that have been created
as a result of nightly processing, as well as any objects that may have been
deleted (which might occur when a difference source originally used as the
basis for a new object is re-classified as an observation of a moving object).
Chunk files are read in their entirety, as are delta files; delta files are
written in their entirety (in the store phase). The spatial sizing of chunks
is determined by two competing factors: using large chunks results in a
smaller number of relatively large and efficient IOs during the load phase,
small chunks better approximate the field of view and so require less total
IO, but result in smaller, less efficient requests. The load phase finds
all chunks overlapping the field of view, reads objects from the corresponding
files, and finishes by spatially indexing these objects into a data structure
that allows for fast spatial matches.

The compute phase waits for a trigger event from the image processing and
detection pipeline; once received, all difference sources for the current
visit are read into memory. For each difference source, all objects within a
fixed distance $R$ are found and stored in the nightly processing database as
difference source/object id pairs. A trigger event from the NightMOPS
pipeline is then awaited; predicted positions for moving objects are read from
the database once it arrives. Next, difference sources within moving object
error ellipses are found and stored in the database, again as difference
source/moving object id pairs. Finally, the difference sources to generate new
objects from are identified using criteria discussed in more detail below; ids
for these sources are also stored in the database.

The store phase of the association pipeline is in charge of updating
historical knowledge of the sky. This entails writing out delta files for
chunks overlapping the FOV. A series of SQL statements operating on the
outputs of all the pipelines are then executed. These scripts:

\begin{itemize}
\item update the observation count and last observation time for matched
      objects in the \code{Object} catalog.

\item insert new objects into the variable \code{Object} catalog

\item set the \code{objectId} field of difference sources to the id of the
      object generated from the source, or to the id of the closest matching
      object.

\item append difference sources from the IPSD output tables to the global
      \code{DIASource} table

\item \emph{(optionally)} append per-visit moving object predictions and
      association results to global tables

\item \emph{(optionally)} drop per-visit pipeline output tables (from IPSD,
      NightMOPS, and AP)
\end{itemize}

\subsubsection{Performance}

Due to the similarity in AP architecture between DC2 and DC3a, the AP
performance profile remains broadly similar between the two data challenges.
We note that the computational and IO burden during the load phase increased
for DC3a due to proper motion correction of object positions (see below);
however, unlike in DC2, the AP was compiled with optimization (\code{-O3}) and
was configured to read chunk files from a parallel filesystem (lustre)
rather than NFS. As a result, the load phase was observed to complete in under
1 second for the runs analyzed (artificially low timings for visits after the
first were ignored: DC3a runs repeatedly visited the same field, and chunk
files were therefore cached almost immediately at the OS/filesystem level).
These timings are roughly equivalent to DC2.

One AP performance concern raised as a result of DC2 performance analysis was
the speed with which difference sources were read from the database in the
compute phase - we had observed timings of \ensuremath{\sim}0.12 seconds for
every thousand difference sources read in. For DC3a, database IO speed was
very significantly improved: it is now performed via a thin LSST wrapper for
the native MySQL API, rather than through a debug build of the CORAL database
access library. As a result, difference sources are read more than an order of
magnitude faster, at \ensuremath{\sim}0.01 seconds per thousand difference
sources. Timings for the rest of the compute phase are comparable to DC2
results.

During the store phase, we observed somewhat more consistent delta file write
timing within a run (0.1 - 0.23 sec, 1 - 1.5 sec); however, wide swings in
timing (for very similar delta file sizes) across runs remain. We hypothesize
this is due to filesystem contention. On the other hand, SQL script execution
times for the runs we examined were much more consistent. Their execution
consumed slightly less than 1 second on average with no significant spikes; in
DC2 timing swings of several seconds within a run were observed.

\subsubsection{Changes for DC3a}

TODO: describe proper motion correction, operation on difference source pairs,
source characterization

\subsubsection{Future Work}

Features to be implemented and/or investigated in the DC3b timeframe include:
\begin{itemize}
\item \emph{Use of inter-slice communications primitives}:
    During DC3a, the pipeline framework was extended with the ability to
    perform inter-slice communication. This allows an implementation of AP
    that does not require a custom inter-slice communication scheme (i.e.
    shared memory), greatly simplifying the code.
\item \emph{Expanded Python Interface}:
    A simplified version of the matching algorithms used by AP have been made
    available in the applications framework; these could be expanded if
    necessary. For DC3b, we also expect to provide a more flexible and python-
    centric version of the AP.
\item \emph{Probabilistic Matching}:
    Positional error ellipses for both objects and difference sources are
    available. Given a difference source matching multiple objects, these
    can be used to pick the best matching object according to probabilistic
    criteria (rather than simply picking the closest one). Alternatively
    or in addition, all difference source to object matches could be
    officially stored for later perusal.
\item \emph{Correction for Refractive Effects}:
    Color dependent refractive atmospheric effects are expected to have a
    significant impact on object positions. Using stored colors for each
    object and the zenith angle for the visit, we expect to be able to
    meaningfully improve the object positions used as input to the spatial
    match routines.
\item \emph{Transactions and Data Volume}:
    We hope to investigate the feasibility of updating the database and chunk
    files in transactional fashion with visit-level granularity, so that from
    a database perspective each visit appears to have either completed
    successfully or not at all. We also plan to run tests in which the input
    \code{Object} catalog is significantly larger than the one used for DC2
    and DC3a; the goal will be to stress the I/O intensive portions of the AP
    with a particular focus on more fully characterizing the performance of
    the store phase.
\end{itemize}

