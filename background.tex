% Section 6.2: Background Subtraction

\subsection{Background Subtraction}
\label{secBackground}

% intro 

A small number of new classes were developed to perform photometric
background estimation.  A \code{Statistics} class was written to compute a
variety of standard statistics (mean, median, standard deviation,
etc.) for the pixel values in an \code{Image} or \code{MaskedImage}.  The classes
\code{Interpolate} (base) and \code{SplineInterpolate} (derived) were written to
provide code to perform a natural spline interpolation over x,y pairs,
and a \code{Background} class, which uses the \code{Statistics} and \code{Interpolate}
classes, was written to estimate the photometric background for use in
sky subtraction.

% philosophy 

The basic philosophy in the algorithm is that an image is {\itshape
relatively} sparsely populated with astrophysical sources, and that the
majority of the pixel values are random variates about the local mean
sky.

% break up the image into a grid of sub-regions
% get the 3-sigma clipped median in each sub-region (cite NR for fast median)
% --> these values should be rough approximations to the sky in each grid cell

Local sky levels were estimated in an image by first subdividing the
image into a grid of subimages.  In each grid cell, a
3$\sigma$-clipped\footnote{Clipping was symmetric, and thus unbiased
for a symmetric distribution such as a Gaussian.} mean was computed to
represent the sky level.  Sky levels will often be contaminated with
stellar flux from objects which are below the detection threshold, and
a median would be a more robust measure of the central sky value under
these circumstances.  However, as such sub-threshold sources will also
contribute to the detected stellar PSFs, their biasing influence is
statistically desirable in an estimate of the background -- they are a
part of the background!  For this reason, a clipped mean was chosen
over a median.  The clipped mean for each cell was assigned to the
row,column coordinate of the cell's central pixel. 

% use a bicubic spline to construct a background image (site NR for bicubic spline)
% - spline along each 'column' of medians to interpolate all pixels in the column
% - spline across each row using the values from the columns

The resulting grid of clipped means was interpolated to create a
background image with the same dimensions as the original.  A bicubic
spline interpolation algorithm was used to mesh the grid smoothly over
the image.  For our bicubic interpolation, the
grid values were first splined along the columns to produce smooth
pixel values along each grid column.  The full background image was
then produced by splining the interpolated column values across each
row.  The first interpolation (along grid columns) is far less
computationally intensive than the second (across {\itshape all}
rows).  It would be equally valid to perform the first interpolation
across the grid rows and the second along all columns.  However, the
row-major (i.e. natural C/C++) structure of the LSST images made it advantageous to adopt
the former option.

% tests

The background estimation code was successfully tested on images with
known background levels: (1) all pixels set to a constant value, (2)
pixel values increasing linearly across the frame, and (3) an image of
noise having known statistical properties.  The tests are included
with the source code.

% not-a-knot

The {\itshape natural}\footnote{I.e. vanishing second derivative at the end knots}
spline algorithm used for interpolation can
suffer from ringing near the edges, and it is our intention to
continue development by implementing a more robust {\itshape
not-a-knot} spline in its place.
