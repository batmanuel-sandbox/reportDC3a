% Section 8: Conclusions

\section{Conclusions}

We draw the following conclusions from the results of Data Challenge 3a.

\begin{itemize}

\item Production runs show improvements in DC3a in execution time for common
stages of the IPSD pipeline when compared to DC2. Additional improvement of
execution time is not scoped for DC3b, but will be a major priority for DC4. We
cannot assume that Moore's Law will solve the performance problem for us. We
continue to believe that the processing times necessary for full production
throughput during real-time operation will be reached, but recognize that this
will require considerable attention in DC4.

\item Source detection now is hampered by the quality of the image subtractions.
Registration errors between the calibrated exposure images and the corresponding
template need to be improved; currently misalignment leads to artifacts which
result in many false source detections, along with very strong spurious peaks in
the resulting lightcurves. DC3b will include improvement of the science data
quality as a major focus.

\item Tests on the Abe cluster showed that the pipelines scaled reasonably 
well when applied to the entire focal plane of a CFHT-LS exposure. The 
pipeline stage harness supporting the science algorithms was designed to
be computationally lightweight, and analysis of runs on Abe show that 
comparatively little processing time is spent by this infrastructure.

\item Early in DC3, the code tree underwent significant refactoring and was
additionally upgraded for 64-bit support. These steps were important and
necessary, but left developers without the ability to test implementations
inside the harness framework until comparatively late in the DC3a development
cycle. DC3b will use several strategies, including continuous integration
and additional tools for running science stages independently from the 
LSST cluster.

\end{itemize}