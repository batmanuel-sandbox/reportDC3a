% Section 6.3: Source Detection

\subsection{Source Detection and Measurement}

\subsection{Classes}

The namespace lsst::afw::detection contains the framework classes related to 
processing astronomical images. The namespace lsst::meas::algorithms contains
the algorithmic implementations for processing detections.
The fundamental objects are:\begin{itemize}
\item Source detection (\code{lsst::afw::detection})
\begin{description}
    \item[Threshold] A threshold level.
    \item[Footprint] A set of pixels (internally stored as Spans).
    \item[DetectionSet] An STL collection of \code{Footprint}s associated 
        with a \code{MaskedImage}.
    \item[FootprintFunctor] Utility for applying a function over the pixels of
        a \code{Footprint}.
\end{description}
\item Source measurement (\code{lsst::meas::algorithms})
\begin{description}
    \item[MeasureSources] Collection of centroid, shape, and photometry 
        measurement algorithms to apply to a \code{Footprint}.
\end{description}
\item Miscellaneous
\begin{description}
    \item[lsst::meas::algorithms::PSF] PSF representation, a base API for
        PSF should be moved to afw in DC3b/DC4
    \end{description}
\end{itemize}

All of these classes are written to make use of image, utility and middleware
framework classes such as \code{lsst::pex::logging::Trace}. They are 
documented using doxygen, although the level of documentation is patchy.

\subsection{Source detection}

The fundamental class in source detection is a \code{Footprint}, a set of 
connected pixels above (or below) some threshold. Two pixels are said to be 
connected if they touch along a side or by a corner. The internal 
representation of a \code{Footprint} is an STL container of \code{Spans} - 
a triple of a row number in the image and a starting and ending column. The 
user does need to be aware of this and can use a \code{FootprintFunctor} to
iterate over the pixels in the \code{Footprint}

A \code{Footprint} or set of \code{Footprint}s only makes sense in the 
context of an image; a \code{DetectionSet} associates an STL container of 
\code{Footprint}s with a \code{MaskedImage}, and may be constructed from 
the \code{MaskedImage} and a \code{Threshold}; it is templated over the 
\code{Image} and \code{Mask} pixel types.

A \code{Threshold} may be constructed to be positive or negative value in 
units of flux, variance or standard deviation. The question of what value to 
adopt for the threshold (although interesting) has thus been abstracted from 
the algorithms and is answered separately by \code{Policy}. 

In DC3a we detect on both difference images and on stacked images, which are 
first smoothed using \code{afw::math::Kernel}. In either use case, the output
is a \code{DetectionSet}.

\subsection{Source Measurement}

Source measurement occurs only in the context of an image, thus 
\code{MeasureSources} - a set of centroid, shape, and photometry measurement 
algorithms - is constructed from an \code{Exposure}; it is templated over 
\code{Exposure}'s pixel type. \code{MeasureSources}  acts as the driver behind
the generalized \code{Footprint} iterator \code{FootprintFunctor} in applying 
its set of algorithms to a \code{Footprint} over the \code{Exposure}. 
Which algorithms to use can be specified by name in a \code{Policy} file.


