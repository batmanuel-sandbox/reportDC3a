% section 1: Introduction

\section{Introduction}
Data Challenge 3 (DC3) is the third in a series of prototypes of the LSST Data Management
System (DMS). Through these data challenges, we seek to identify the most challenging technical
problems to building a DMS that meets the LSST science goals. We prototype specific solutions to
these challenges with the expectation that by the start of the construction phase of the telescope,
we will have a well-defined plan for how to build a DMS that can perform at the level needed by
first light. Despite the prototyping nature of the data challenges, we are not producing throwaway
code; rather, we expect that the software we produce in the data challenges will serve as the
foundation for the DMS that will be completed during the construction phase.
In Data Challenge 2 (DC2) \iffalse \ref{DC2report}\fi, we focused on demonstrating the use of
astronomical algorithms for nightly processing in an LSST processing
framework.  The scope of DC3 is significantly larger, including both a
more capable implementation of the Alert Production and a first
prototype implementation of the Data Release Production.

Because of the large scope of DC3, we have divided it into two phases,
DC3a and b.  DC3a includes only the Alert Production capabilities,
while DC3b adds the Data Release Production.  This report describes the results of 
DC3a only. We enumerate our goals, summarize the implementation, and report on what we've learned from it.

\subsection{Goals of DC3a}

The goals of DC3a were in part an outcome of the DC2 post-mortem
analysis.  The DC2 report listed several areas where improvement was
needed, and these have largely been incorporated into the DC3a goals,
which are as follows:

\begin{itemize}
\item Application Framework improvements
\item Middleware improvements
\item Implement the Instrument Signature Removal (ISR) pipeline
\item Implement the Image Characterization Pipeline (ICP), in
  particular the determination of the WCS.
\item Implement an initial SDQA system
\item Improve the science quality of the difference images and
  resulting catalogs
\item Improve the execution speed to a level that gives confidence in
  scaling to the full LSST
\item Improve the capabilities of the software build system
\end{itemize}

\subsection{Metrics and Validation of DC3a}

For DC3a, we have instituted a more quantitative set of metrics to
evaluate the outcome than we used for DC2.  They are as follows:

\begin{itemize}
\item UML model completeness - more generally, SQA metrics
\item Visit latency
\item Fraction of input images that complete pipeline processing
  without software crashes or exceptions
\item Fraction of input images with a successful WCS determination
  (registration to template image better than 0.5 arcsec)
\item Fraction of successfully subtracted images (RMS noise over whole
  image $\leq$ 1.5*sky; RMS of residuals around bright isolated stars
  $\leq$ XX*peak unsubtracted value)
\end{itemize}

The achieved values of these metrics are presented in Section \XXX{metrics}.
