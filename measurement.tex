\subsection{Source Measurement}
\label{secSourceMeasurement}

Measuring Sources' properties is an open-ended task;  not only is the
number of properties (centroid, shape, psf-photometry, galaxy-photometry,
bulge-disk decomposition, ...) open ended, but the algorithm to be
used for each property is the subject of active research, or at least
controversy.  DC3a therefore uses a flexible driver class for measuring
sources, \code{MeasureSources}.  The class knows which \textit{properties}
should be measured (currently centroid, shape, and photometry), but
it knows nothing of which \textit{algorithms} should be employed.  

\subsubsection{Source Measurement Factories}
\label{secSourceMeasurementFactories}

Because the code that actually measures things needs to look at pixel
values, the classes that do the work are templated on the image type
and contain a method, \code{apply}, that takes an image, a position,
and any other required information (e.g. a \code{Psf}) and return the
desired information.

For example, centroiding is achieved using \code{MeasureCentroid<MaskedImageT>} and\break \code{MeasureCentroid::apply}
returns a \code{Centroid} (with members such as \code{getXErr()}).  \code{MeasureCentroid} itself is a Singleton which
maintains a registry of available algorithms.  These algorithms are contained in separate source files, and register
themselves by initialising a variable at file scope; this registration associates a string (e.g. \code{"NAIVE"}) with a
class (e.g. \code{NaiveMeasureCentroid}).

Once this registration is achieved, \code{MeasureSources}'s constructor can look up algorithms by name (e.g. in a policy
file) even if the algorithmic code (e.g. \code{BestMeasureCentroid}) was loaded dynamically.

\subsubsection{Centroiding}

Two centroiding algorithms are available for DC3a, a simple unweighted first moment
of the 3\by3 region around a pixel (\code{"NAIVE"}) and a simplification of the algorithm employed by
SDSS (\code{"SDSS"}, \citet{SDSSAstrom}).

The SDSS algorithm involves smoothing the object with an approximation to the PSF, then examining
the 3\by3 region around the peak in the smoothed image.  We fit a quartic approximation to
a Gaussian to the 3 ``vertical'' sets of 3 pixels, determining a maximum in each --- these three
maxima define a line.  We then repeat this operation for the 3 ``horizontal'' sets of 3 pixels,
defining another line.  The intersection of these two lines defines the centroid.  Unlike the
SDSS code, we do not attempt to debias this centre for the known shape of the true PSF at this
point in the frame.

It is not clear that this approach delivers the combination of accuracy and robustness that will be required to meet
LSST's astrometric requirements.

\subsubsection{Shape Measurement}

Only one algorithm is available for DC3a, the adaptive moment algorithm
employed by the SDSS.  The idea is to measure
$$
M_{nm} \equiv \int x^n y^m I(x, y) w(x, y) \,dx\,dy
$$
where $nm = \{20, 11, 02\}$ and $w(x, y)$ is a 2-dimensional Gaussian weight function.  The
parameters of this Gaussian are then adjusted until the measured
moments, $M_{nm}$, are consistent with $I$ and $w$ having the same second moments.

In the context of weak lensing, these $M_{nm}$ can be used to estimate shapes providing
that they are also measured for the PSF at the position of the object, and a fourth
moment is also measured to allow for polarisability;  neither of these are done in the
DC3a implementation.   We do not expect to use these adaptive moments for lensing, but
they are used to e.g. choose candidate PSF stars (\Sec{secChoosePSFStars}).

\subsubsection{Photometry}

\XXX{Write me}
