% section 5.1.3: Pipeline Orchestration

\subsubsection{Processing Orchestration and Control}

Closely related to pipeline execution packages are the control
packages ({\tt ctrl\_}).  The {\tt ctrl\_events} package provides the
event framework that allows pipelines to talk to each other.  The {\tt
ctrl\_evmon} provides the event monitor services which can listen to
events (particularly logging events) and react intelligently.  The
{\tt ctrl\_orca} realizes the {\it orchestration layer} responsible
for launching pipelines.  An important job this layer must do is
record {\it provenance} information--all of the data describing the
software environment and policy paramters that was used to configure
and execute a pipeline.  

\subsubsubsection{Pipeline Orchestration} 
\label{sec:PipelineOrchestration}

In DC2, the orchestration layer was largely undeveloped; in lieu, we
had a customized launching script that handled several of chores that
in general would be handled by the orchestration layer.  In DC3a, we
used this script as a starting use case to design a more generalized
layer configuring and launching pipelines.  The design also included
the notion of monitoring a pipeline as it runs, but we did not
implement that part in DC3a.  

One of the challenges we want the layer to address is adapting a
pipeline to run on different platforms.  These platforms may differ in
terms of, say, where data can be read in from and written out to, how
the pipeline must be launched--e.g. right away via a remote shell
command (as with our development cluster) versus asynchronously via a
batch scheduler (required for NCSA Abe), and how one checks on its
progress.  We also want it to be able to launch any kind of
pipeline--not just one based on our pipeline harness.  To address
these challenges, the {\tt orca} design segregates its various duties
into separable classes, allowing for specialized versions to be
plugged in to handle different types of pipelines and platforms.
Similar, we had to develop a policy configuration that could mirror
the class model and allow, for instance, the specialized classes to be
specified in policy.  

The main duties of the orchestration layer are:
\begin{itemize}
\item Create working directories on the target platforms for input and
  output data, policy files, and log files,
\item Initialize the database tables to be used by the pipelines, 
\item Deploy all needed policy files and launch scripts onto the
  target platforms,
\item Set the necessary software environment, 
\item Record provenance data
\item Remotely execute each platform launch script.
\item Monitor the pipeline for failures (not implemented in DC3a)
\item Shut down the pipeline (not implemented in DC3a)
\end{itemize}

The main classes responsible for this are:

\begin{description}
\item[\tt ProductionRunManager:]  a class for configuring and launching the
set of pipelines that make up a production run.  
\item[\tt PipelineManager:]  a class for configuring and launching a
specific pipeline.  We had two implementations of this abstract class:
{\tt SimplePipelineManager} for launching on our development cluster,
and {\tt AbePipelineManager} for launching on the NCSA Abe machine.  
\item[\tt DatabaseConfigurator:]  a class use by a {\tt
ProductionRunManager} and/or a {\tt PipelineManager} (depending on the
policy configuration) to set up the database tables required by the
pipelines.  
\item[\tt Provenance:]  used by the {\tt ProductionRunManager} to
record the provenance data--namely, the policy data--into the
database.  
\end{description}

It's worth noting that when launching a pipeline, our {\tt
PipelineManager} implementations ultimately end up launching a
pipeline launch script through a remote shell.   This launch script is
one that could be run directly by a user (say, for debugging purposes)
independently from the orchestration layer.  In other words, a
pipeline has no dependencies on the orchestration layer to run; this
is a useful feature of the orchestration layer design.  A consequence
of this is that the policy that will configure the pipeline must be
fully specified, and this must include parameters that would be more
conveniently set at the production level.  An example is the {\tt
eventBrokerHost} parameter which sets the host where the event broker
is running; since the broker is responsible for routing event messages
between pipelines, all pipelines need to use the same broker.  Thus,
to ensure consistency, we want to specify it once for the whole
production, but it is used by each pipeline, and so the pipeline
policy needs to have this information.  

We address this dilemma by allowing the parameter to be set at both
the production level {\it and} the pipeline level.  The latter will be
considered the default value that will get used when a pipeline is
launched on its own, independently from the orchestration layer.
However, when the pipeline is launched from orchestration, the
production level value will override the pipeline level value.  The
way this is accomplished is that before the orchestration layer
deploys the pipeline policy files onto the (remote) pipeline platform,
it collects all of the share policy data and rewrites the pipeline
level policy, overriding the defaults.  This fact has important
implications for provenance.  


\subsubsubsection{Provenance}

The orchestration layer (described above) in DC3a generates enhanced
provenance information.  In particular, the software environment and
the contents of the policy files used to run the production are
written both to log files and to the database.  Recording the software
environment allows the exact configuration of LSST-packaged software
used for the run to be reproduced later, while recording the policy
files captures both platform configuration information such as the
compute nodes and database used as well as all configurable science
algorithm settings.

This provenance information, in combination with an event sent to the
production, is sufficient to enable accurate reconstruction of a given
data product resulting from that event, although a demonstration of
automated reconstruction was deferred.  When combined with the full
sequence of events sent to the production, the provenance allows exact
duplication of a given run.  The recorded provenance proved highly
useful while debugging algorithmic issues since it simplified the
construction of small reproducible test cases demonstrating problems.

The software environment is characterized by the versions of packages
maintained by {\tt eups} that are ``setup'' at the time of production
execution (see DC2 report for more details).  In addition, the actual
directories declared as the installation locations of the packages are
also persisted, allowing locally-setup packages and packages installed
under {\tt \$LSST\_DEVEL} to be identified.

The recorded policy file information includes the contents on a per-key
basis as well as an MD5 hash of the file contents and the file's
last-modified-time.  These latter two items are intended for eventual
use to remove duplicate entries when policy files are reused across
multiple runs.

The provenance written to the database goes into two sets of tables: one
set in the per-run database and one in a global database ({\tt
DC3a\_DB}) that spans all DC3a runs.  The global database permits
queries to find runs that used a given configuration.  For example, this
query finds all runs that had the {\tt pixelScaleRangeFactor} set to a
number other than 1.1:

\begin{verbatim}
SELECT prv_Run.runId, prv_PolicyKey.keyName, prv_cnf_PolicyKey.value
FROM prv_Run, prv_PolicyKey, prv_cnf_PolicyKey
WHERE prv_PolicyKey.policyKeyId = prv_cnf_PolicyKey.policyKeyId
  AND keyName = 'pixelScaleRangeFactor'
  AND value != '1.1'
  AND FLOOR(prv_PolicyKey.policyFileId / 65536) = prv_Run.offset;
\end{verbatim}

Similarly, this query finds all runs that used version 3.0.9 of the {\it
meas\_algorithms} package:

\begin{verbatim}
SELECT runId
FROM prv_SoftwarePackage NATURAL JOIN prv_cnf_SoftwarePackage
     JOIN prv_Run ON (FLOOR(prv_SoftwarePackage.packageId / 65536) = offset)
WHERE packageName = 'meas_algorithms' AND version = '3.0.9';
\end{verbatim}

