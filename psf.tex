% Section 6.4: PSF Determination

\subsection{PSF Determination}
\label{secPsf}

\subsubsection{Classes}

The PSF virtual base class, \code{Psf}, is currently in the
\code{lsst::meas::algorithms} package, but should be moved
to \code{lsst::afw} for DC3b.

\code{Psf} supports a PSF that is a function of position.  
In addition to constructors and destructors, the class interface
permits convolving an image with the PSF or evaluating a desired
pixel (or returning an \code{Image}) of the PSF at a given position in a frame.

There are two concrete \code{Psf} classes in DC3a: \code{dgPsf} (a circularly-symmetric double Gaussian) and
\code{pcaPsf} (a linear combination of basis functions, modelled after SDSS's PSF representation).  Both of these
classes are instantiated using a helper class, \code{PsfFactory}, that takes a string argument to identify the desired
type of PSF; this enables us to add a new PSF representation without modifying the base class.\footnote{In fact, the new
  PSF class could be separately compiled and loaded at run time using python's \code{import} command.}  The implementation of \code{PsfFactory} is similar to the factories employed in Source Measurement (\Sec{secSourceMeasurementFactories}).

\subsubsection{Spatial Modelling}

In modelling a PSF's spatial structure, we need to ensure that the PSF is well modelled at all
points in the image, rather than just where we happen to have bright isolated stars.  It is
therefore important to use stars which are as uniformly as possible distributed, even if this
means that we have to accept fainter stars than we might like.

A given image is divided up into cells, with each cell represented by an instance of \code{SpatialCell}. These cells
each contribute only a limited number of objects to the spatial fit.  The \code{SpatialCellSet} class maintains all of
these cells as an STL collection of \code{SpatialCell}s.  Each cell itself contains an STL collection of instances of
classes derived from \code{SpatialCellCandidate} (e.g. the PSF stars themselves).

\code{SpatialCellSet} uses the Visitor pattern \citep{GoF} to allow the user to apply a functor to the `best'
\code{SpatialCellCandidate} in each \code{SpatialCell}, estimating a spatial model.  The user can then go through the
candidate lists rejecting or approving each object; if needs be the spatial model can then be improved.  If all
instances in a list are rejected from the spatial model, the least-bad one will be used.

A superclass of \code{SpatialCellCandidate}, \code{SpatialCellImageCandidate}, contains an image
of some sort; in the case of PSF determination, this is candidate PSF star, but the
same framework could be used in \code{ip\_diffim} in which case the image would
be a difference image kernel.

\subsubsection{Choice of PSF stars}
\label{secChoosePSFStars}

The standard LSST detection algorithm (\Sec{secDetection}) is run to find a set
of candidate objects, followed by the measuring the objects' positions and
second moments (\Sec{secSourceMeasurement}).  We then form a 2-dimensional
histogram (i.e. an image) from the $M_{20}$ and $M_{02}$ (i.e. \code{xx} and \code{yy}) moments
and find \textit{its} maximum and second moments;  objects close to this maximum
are taken as our initial candidate list for stars to measure the PSF.

\subsubsection{PSFs as sums of PCA components; \code{pcaPsf}}

The DC3a PSF is based on a \code{lsst::afw::math::LinearCombinationKernel}, where
the components are the eigen-images derived from a set of stars in the image (cf. \citet{photoADASS});
we are not wedded to this representation post-DC3a.

In terms of the previous section,  we first find a set of \code{Source}s in an image
that may be good isolated stars,  and insert them into a \code{SpatialCellSet}.  The
routine \code{createKernelFromPsfCandidates} visits each cell and performs the PCA
to find the desired eigen-images using a Visitor and the \code{lsst::afw::image::ImagePca} class.
Given this kernel, \code{fitSpatialKernelFromPsfCandidates} uses a different
Visitor to evaluate the $\chi^2$ of the fit, and then \code{minuit} to find
the best-fit spatial variation (modelled in terms of a 2-dimensional polynomial).  Finally,
we call the \code{createPSF} factory function to generate a \code{Psf}.
In semi-pseudo-code this looks like:

\begin{center}
  \begin{minipage}{13cm}
    \small
\begin{verbatim}
mi = MaskedImageF()
cellSet = SpatialCellSet(BBox(PointI(0, 0), width, height), 100)

for source in sourceList:
    cellSet.insertCandidate(makePsfCandidate(source, mi))

kernel, eigenValues = \
       createKernelFromPsfCandidates(cellSet, nEigenComponents, spatialOrder,
                                     kernelSize, nStarPerCell)

fitSpatialKernelFromPsfCandidates(kernel, cellSet, nStarPerCellSpatialFit, tol)

psf = createPSF("PCA", kernel)
\end{verbatim}
  \end{minipage}
\end{center}
We have omitted the rejection of bad candidates in the interest of clarity.
