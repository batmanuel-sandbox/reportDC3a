% section 5.2: Database Schema Modifications

\subsection{The Database}

DC3a builds on the basic database schema design set down in DC2
(consisting of 23 tables) which are realized in a MySQL database
engine running on one of the nodes of our development cluster.  The
design is tracked in our UML diagram, and care has been taken that the
information model mirrors the software class model where appropriate.
Pipeline operation relies on two databases.  First is a global set of
tables that were created once and reused with every production run.
The second consists of a set of tables that are created for a
particular production run (and are given a database name that includes
the run identifier; see section \ref{sec:components}); these contain
the actual results of a particular run.  When the run is done, this
database may be deleted if the results are no longer desired.  

The following subsections we describe how the database use has evolved
in DC3a.  

\subsubsection{The Database Schema}

The following (relatively minor) changes were made to the schema for DC3a:

\begin{itemize}

\item The MOPS schema was integrated into the master copy of the LSST
  baseline schema in Enterprise Architect.  The integration included
  realigning names, units, and merging tables where appropriate.

\item Amplifier-level exposure metadata was introduced, in addition to
  the CCD-level and FPA-level metadata already present in DC2.

\item An SDQA-related schema was introduced; the details of this
  schema are covered below in Subsection~\ref{sdqasubsection}.

\item A schema for capturing provenance related to policies was
  introduced (see also section \ref{sec:provenance}).

\item The whole schema was refactored to clean up types and realign it
  with LSST naming conventions. 

\item The schema for the Object, Source and DIASource tables was
  modified to reflect the new needs of the application classes for
  astronomical-object transients. 
\end{itemize}

\subsubsection{Schema browser}

In DC3a we deployed an on-line schema
browser\footnote{http://dev.lsstcorp.org/schema/} to enable broad
visibility (particularly for scientists) into our schema.  This
graphical, web-based browser allows a quick, easy and convenient way
to look at the database-schema structure, including browsing tables,
column names, their types, units, UCDs, and descriptions.
Descriptions and units for most columns were added, but the UCD fields
remain empty.

To ensure that the browser reflected our latest official design, its
data is derived directly from the UML model.  This required first that
the UML model be expanded to capture information about per-column
units, UCDs and descriptions.  A script then translates this
information into a format readable by the schema browser.  

\subsubsection{Operational Issues}

As a security measure, database authorization was reworked.  In DC3a,
the run-specific databases are created under the ownership of the user
running the pipelines.  To enable automatic authorization during
pipeline operations, the user must (prior to launch) set-up a special
database configuration file in his or her home directory (under an
{\tt .lsst} directory) which contains the username and password to
use.  We also created a tool for granting lsst-specific privileges to
new users as well as scripts for deleting run-specific databases that
are no longer needed.

In DC3a, we also introduced metadata about runs. This allows us to
easily related databases with runs, and cleanup old or obsolete runs
based on different criteria such as creation time or run owner.

Database setup scripts, used by the orchestration layer, were
reworked. A thin layer between the database and the admin scripts was
introduced to contain the code directly interacting with the
database. This will simplify migration to other database technology,
should this become necessary in the future.  The database setup scripts
were also integrated with policies.




