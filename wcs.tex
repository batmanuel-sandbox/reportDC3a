% Section 6.5: WCS Estimation

\subsection{WCS Determination}

Wcs, or World Coordinate System is a convention for mapping positions on an astronomical image to coordinates in the sky. We require that our Wcs map our positions with an accuracy of better than 0.1", driven by the need to subtract images from templates in the {\tt diff_im} stage. For DC3a we implemented a method of determining a Wcs solution for an image using the {\tt astrometry.net}\footnote{http://astrometry.net} package written by Dustin Lang. 

{\tt astrometry.net} takes as input a list of positions pixel positions of sources in an image, and compares pattern of stellar positions, or asterisms, with a template list of stars with known right ascension and declination. It can be used to match images of arbitary pixel scale, location on sky, and magnitude limit, provided there is a sufficient overlap between the set of sources in the image, and those in the template. {\tt astrometry.net} comes supplied with a large template based on the USNO-b catalogue, but we found this template unsatisfactory for two reasons. The sheer size of the cataloge ($\sim$30\,Gb) meant that disk I/O was prohibitively expensive. We also found that the centroids of individual objects differed by up to 0.3" between the CFHT images and the positions measured by USNO-b. Instead we created a smaller template (2\,Mb) from images of the CFHT field created by Swarp, and found that method worked to our satisfaction.

\subsubsection{Improvements to the Wcs class}
Astronomical images generally contain some distortion, i.e the rectangular grid of pixels in an image does not map onto a retangular grid of right ascension and declination. The Wcs standard does not supply a method of dealing with this real world problem. {\tt astrometry.net} describes distortion coefficients using the SIP convention (Shupe et al. 2005; ASPC 347 491), which was developed and used the the Spitzer Science Center. SIP defines a set of polynomials used to map the positions of the distorted pixels into an undistorted space where they can be correctly transformed using the Wcs. {\tt afw} adopted the SIP convention, and the Wcs class was extended to use the SIP polynomials when converting from pixel to sky coordinates.


\subsubsection{Future Work}
Templates have been produced for the 4 CFHT deep fields. However, the proceedure for producing these templates is quite involved, and needs to be simplified so that templates can be produced on demand for new fields.
